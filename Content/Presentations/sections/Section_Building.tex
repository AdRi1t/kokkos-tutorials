%!TEX root = ../modularized/KokkosTutorial_01_Introduction.tex
% \begin{frame}{DOE ECP Acknowledgement}

% \textit{
% This research was supported by the Exascale Computing Project (17-SC-20-SC),
% a joint project of the U.S. Department of Energy’s Office of Science and National Nuclear Security Administration,
% responsible for delivering a capable exascale ecosystem, including software, applications, and hardware technology,
% to support the nation’s exascale computing imperative. 
% }

% \end{frame}

%==============================================================================

\begin{frame}[fragile]


  \vspace{10pt}
  {\Huge Building Applications with Kokkos}

  \vspace{10pt}

  \textbf{Learning objectives:}
  \begin{itemize}
    \item{Install Kokkos via CMake}
    \item{Build Kokkos inline via CMake}
    \item{Using Spack}
    \item{Build Kokkos inline via GNU Makefiles}
  \end{itemize}

%  \vspace{-20pt}
  \pause

  \begin{block}{Ignore This For Tutorial Only}
     The following details on options to integrate Kokkos into your build process are NOT necessary to know if you just want to do the tutorial.
  \end{block}

\end{frame}

\begin{frame}[fragile]{Options for Building Kokkos}

\begin{itemize}
\item \textbf{Install Kokkos via CMake:} For large projects with multiple dependencies installing Kokkos via CMake and then building against it is the best option.
\item \textbf{Build Kokkos inline via CMake:} This is an option suited for applications which have few dependencies (and no one depending on them) and want to build Kokkos inline with their application.
\item \textbf{Using Spack:} For projects which largely rely on components provided by the Spack package manager.
\item \textbf{Build Kokkos inline via GNU Makefiles:} The option for projects which don't want to use CMake. Only inline builds are supported via Makefiles though. Often this works well for small applications, with few if any dependencies. 
\end{itemize}
\end{frame}

\begin{frame}[fragile]{Kokkos CMake Basics}
\begin{itemize}
\item In the spirit of C++ for \emph{code} performance portability, modern CMake aims for \emph{build system} portability
\item Projects that depend on Kokkos should be agnostic to the exact build configuration of Kokkos
\item No CUDA details in C++! No CUDA details in CMake!
\item Single build system call in your project should configure all compiler/linker flags:

\begin{shell}
add_library(myLib goTeamVenture.cpp)
target_link_libraries(myLib PUBLIC Kokkos::kokkos)
\end{shell}
\item Kokkos configure options are enabled/disabled via CMake as:
\begin{shell}
  cmake -DKokkos_XYZ=ON
\end{shell}
\end{itemize}
\end{frame}

\begin{frame}[fragile]{CMake Backend Options}
\begin{itemize}
\item Numerous backends can be activated 
  \begin{itemize}
    \item Only one GPU, one parallel CPU, and Serial at the same time!
  \end{itemize}
\item \inlineshell{-DKokkos_ENABLE_CUDA=ON}
\item \inlineshell{-DKokkos_ENABLE_HIP=ON}
\item \inlineshell{-DKokkos_ENABLE_SYCL=ON}
\item \inlineshell{-DKokkos_ENABLE_OPENMP=ON}
\begin{uncoverenv}<2->
\item Verify execution spaces in CMake Output, e.g. CUDA
\begin{shell}[linebackground={
  \btLstHL{4}{orange!30}
}]
-- The project name is: Kokkos
...
-- Execution Spaces:
--     Device Parallel: CUDA
--     Host Parallel: NONE
--       Host Serial: SERIAL
\end{shell}
\end{uncoverenv}
\end{itemize}
\end{frame}

\begin{frame}[fragile]{CMake Architecture Options}
\begin{itemize}
\item Device backends \emph{require} architecture be specified (CUDA , OpenMPTarget, and HIP)
\begin{itemize}
  \item \inlineshell{-DKokkos_ARCH_VOLTA70=ON}
  \item \inlineshell{-DKokkos_ARCH_AMD_GFX90A=ON}: MI250X
\end{itemize}
\item Host backends \emph{recommend} architecture be specified to enable architecture-specific optimizations
\begin{itemize}
  \item \inlineshell{-DKokkos_ARCH_HSW=ON}: Haswell
  \item  \inlineshell{-DKokkos_ARCH_ZEN2=ON}: Ryzen (2nd gen)
\end{itemize}
\item Architecture flags will automatically propagate to your project via transitive CMake properties
\begin{uncoverenv}<2->
\item Verify architectures in CMake Output, e.g. Volta 7.0
\begin{shell}[linebackground={
  \btLstHL{4}{orange!30}
}]
-- The project name is: Kokkos
...
-- Architectures:
--  VOLTA70
\end{shell}
\end{uncoverenv}

\end{itemize}
\end{frame}


\begin{frame}[fragile]{CMake And CUDA}
\begin{itemize}
\item Kokkos is a \emph{C++} performance portability layer, but CUDA is usually built as a separate language with \inlineshell{nvcc}.
\item \inlineshell{nvcc} doesn't accept all C++ compiler flags
\item Kokkos' solution for now is to provide  \inlineshell{nvcc_wrapper} that converts \inlineshell{nvcc} into a full C++ compiler.
\uncover<2-> { \item Set CMake C++ compiler to \inlineshell{nvcc_wrapper} }
\uncover<3-> { \item CMake will report compiler as host C++ compiler }
\end{itemize}

\begin{uncoverenv}<2->
\begin{shell}[linebackground={%
    \btLstHL<2>{2}{orange!30}%
}]
> cmake ${KOKKOS_SRC}
  -DCMAKE_CXX_COMPILER=${KOKKOS_SRC}/bin/nvcc_wrapper
  -DKokkos_ENABLE_CUDA=ON
\end{shell}
\end{uncoverenv}

\begin{uncoverenv}<3->
\begin{shell}[linebackground={
   \btLstHL<3>{1}{green!30}%
   \btLstHL<4>{2}{orange!30}
}]
-- The CXX compiler identification is GNU 8.2.0
-- Check for working CXX compiler: bin/nvcc_wrapper
\end{shell}
\end{uncoverenv}

\begin{uncoverenv}<4->
\begin{itemize}
  \item Or simply use clang++ as your compiler...
\end{itemize}
\end{uncoverenv}

\end{frame}

\begin{frame}[fragile]{CMake And HIP}

\textbf{Enable HIP backend}
Configure with:
\begin{lstlisting}[language=bash]
-DKokkos_ENABLE_HIP=ON
\end{lstlisting}

\textbf{Compiler}
 Need to explicitly set \texttt{hipcc} or \texttt{amdclang++} as C++ compiler:
\begin{lstlisting}[language=bash]
-DCMAKE_CXX_COMPILER=hipcc
\end{lstlisting}

\textbf{Architecture flags}
Chose one from:
\begin{lstlisting}[language=bash]
-DKokkos_ARCH_AMD_GFX908=ON # for AMD Radeon Instinct MI100
-DKokkos_ARCH_AMD_GFX90A=ON # for AMD Radeon Instinct MI200 series
\end{lstlisting}

\end{frame}

\begin{frame}[fragile]{CMake And SYCL}

\textbf{Enable SYCL backend}
Configure with:
\begin{lstlisting}[language=bash]
-DKokkos_ENABLE_SYCL=ON
\end{lstlisting}

\textbf{Compiler}
 Need to explicitly set \texttt{icpx} as C++ compiler:
\begin{lstlisting}[language=bash]
-DCMAKE_CXX_COMPILER=icpx
\end{lstlisting}

\textbf{Architecture flags}
Chose one from:
\begin{lstlisting}[language=bash]
-DKokkos_ARCH_INTEL_GEN=ON # JIT compiler
-DKokkos_ARCH_INTEL_PVC=ON # for GPU Max 1550/Ponte Vecchio
\end{lstlisting}

\end{frame}

\begin{frame}[fragile]{CMake And OpenMPTarget}
\begin{itemize}
\item Similar configuration as CUDA/HIP backends, but use:
\begin{shell}
cmake -DKokkos_ENABLE_OPENMPTARGET=ON
\end{shell}
\item Still requires target device architecture to be given:
\begin{shell}
cmake -DKokkos_ARCH_VOLTA70=ON
\end{shell}
\item Currently very sensitive to exact compiler/STL combination
\begin{itemize}
\item Clang15+
\item GCC9 Toolchain
\item See \url{scripts/docker/Dockerfile.openmptarget} for recipe
\end{itemize}
\item C++17 is required
\item Working on Spack packages to handle complex version dependencies
\end{itemize}
\end{frame}

\begin{frame}[fragile]{Building Against an Installed Kokkos (i)}

Find exported Kokkos configuration (include dirs, libraries to link against, compile options, etc.)
and generate my project's build system accordingly.

\textbf{Basic starting point}
Create a \texttt{CMakeLists.txt} file.
\begin{lstlisting}[language=bash]
cmake_minimum_required(VERSION 3.16)
project(myProject CXX) # C++ needed to build my project

find_package(Kokkos REQUIRED) # fail if Kokkos not found

# build my executable from the specified source code
add_executable(myExe source.cpp)
# declare dependency on Kokkos
target_link_libraries(myExe PRIVATE Kokkos::kokkos)
\end{lstlisting}

\textbf{Working with a library}
\begin{lstlisting}[language=bash]
find_package(Kokkos 4.0 REQUIRED) # request Kokkos minimum version
add_library(myLib ${SOURCES})
target_link_libraries(myLib PUBLIC Kokkos::kokkos)
\end{lstlisting}

\end{frame}

\begin{frame}[fragile]{Building Against an Installed Kokkos (ii)}

\textbf{Finding Kokkos} Add Kokkos installation prefix to the list of directories searched by CMake:
\begin{lstlisting}[language=bash]
cmake .. -DKokkos_ROOT=<prefix> -DCMAKE_CXX_COMPILER=<...>
\end{lstlisting}

\textbf{Kokkos package introspection}
Assert that support for \texttt{\_\_host\_\_}, \texttt{\_\_device\_\_} annotations in lambdas declaration is enabled
\begin{lstlisting}[language=bash]
# (optional) assume my project uses lambdas
if(Kokkos_ENABLE_CUDA)
  # fatal error if not enabled
  kokkos_check(OPTIONS CUDA_ENABLE_LAMBDA)
endif()
\end{lstlisting}
or query that generation of relocatable device code is enabled
\begin{lstlisting}[language=bash]
kokkos_check(
  DEVICES CUDA
  OPTIONS CUDA_RELOCATABLE_DEVICE_CODE
  RESULT_VARIABLE KOKKOS_HAS_CUDA_RDC)
if(KOKKOS_HAS_CUDA_RDC)
  ...
\end{lstlisting}

\end{frame}


\begin{frame}[fragile]{CMake Building Kokkos Inline}
Build Kokkos as part of your own project (as opposed to finding a pre-installed Kokkos)
\begin{lstlisting}[language=bash]
add_subdirectory(<kokkos source dir>)

# identical as when finding an installed Kokkos package
add_executable(myExe ${SOURCES})
target_link_libraries(myExe PRIVATE Kokkos::kokkos)
\end{lstlisting}

Pass Kokkos options along with app-specific options at configuration time
\begin{lstlisting}[language=bash]
cmake .. -DCMAKE_CXX_COMPILER=<kokkos dir>/bin/nvcc_wrapper \
  -DKokkos_ENABLE_CUDA=ON -DKokkos_ENABLE_CUDA_LAMBDA=ON \
  -DmyApp_ENABLE_FOO=ON -DmyApp_ENABLE_BAR=ON
\end{lstlisting}

\end{frame}

\begin{frame}[fragile]{Kokkos via Spack: Command Line}
\begin{itemize}
\item Spack provides a package manager that automatically downloads, configures, and installs package dependencies
\item Kokkos itself can be easily installed with specific variants (+) and compilers (\%)
\begin{shell}
spack install kokkos@develop +openmp %gcc@8.3.0
\end{shell}
\item Good practice is to define ``best variant`` in your packages.yaml directory, e.g. for Volta system
\begin{shell}
packages:
   kokkos:
    variants: +cuda +openmp +cuda_lambda +wrapper \
              ^cuda@12.0 cuda_arch=70
    compiler: [gcc@8.3.0]
\end{shell}
\item Build rules in \inlineshell{package.py} automatically map Spack variants to correct CMake options
\item Run \inlineshell{spack info kokkos} to see full list of variants
\end{itemize}
\end{frame}

\begin{frame}[fragile]{Kokkos via Spack: Package Files}
\begin{itemize}
\item Build rules created in a \inlineshell{package.py} file
\item Step 1: Declare dependency on specific version of kokkos (3.x, master, or develop)
\begin{shell}
class myLib(CMakePackage):
  depends_on('kokkos@3.2')
\end{shell}
\item Step 2: Add build rule pointing to Spack-installed Kokkos and same C++ compiler Kokkos uses
\begin{shell}
def cmake_args(self):
  options = []
  ...
  options.append('-DCMAKE_CXX_COMPILER={}'.format(
     self.spec['kokkos'].kokkos_cxx)
  options.append('-DKokkos_ROOT={}'.format(
     self.spec['kokkos'].prefix)
  return options
\end{shell}
\item Full details can be found in Spack.md in Kokkos repo.
\end{itemize}
\end{frame}

\begin{frame}[fragile]{Building Kokkos Inline via GNU Makefiles}
	\textbf{Building Kokkos inline with GNU Makefiles in three steps:}

        \begin{itemize}
	        \item Set Kokkos Options e.g. \texttt{KOKKOS\_DEVICES}, \texttt{KOKKOS\_ARCH}
		\item Include \texttt{Makefile.kokkos}
		\item Add \texttt{KOKKOS\_CXXFLAGS, KOKKOS\_LDFLAGS} etc. to build rules
	\end{itemize}

	\textbf{Most Important Settings:}

	\begin{itemize}
	   \item \texttt{KOKKOS\_DEVICES}: What backends to enabled. Comma separated list: \texttt{Serial,OpenMP,Cuda,HIP,OpenMPTarget}
	   \item \texttt{KOKKOS\_ARCH}: Set architectures. Comma separated list: \texttt{HSW,Volta70,Power9,...}
	\end{itemize}

	\pause
	\begin{block}{Order Matters!}
	   Add default target, Kokkos settings, and CXXFLAGS before including Makefile.kokkos!
	\end{block}
\end{frame}

\begin{frame}[fragile]{Example Makefile}
\begin{tiny}
\begin{shell}
KOKKOS_PATH = ${HOME}/Kokkos/kokkos
SRC = $(wildcard *.cpp)
KOKKOS_DEVICES=OpenMP,Cuda
KOKKOS_ARCH = SKX,Volta70

default: test
  echo "Start Build"

CXX = clang++
CXXFLAGS = -O3 -g
LINK = ${CXX}

OBJ = $(SRC:.cpp=.o)

include $(KOKKOS_PATH)/Makefile.kokkos

test: $(OBJ) $(KOKKOS_LINK_DEPENDS)
  $(LINK) $(KOKKOS_LDFLAGS) $(OBJ) $(KOKKOS_LIBS) -o test

%.o:%.cpp $(KOKKOS_CPP_DEPENDS)
  $(CXX) $(KOKKOS_CPPFLAGS) $(KOKKOS_CXXFLAGS) $(CXXFLAGS)  -c $<
\end{shell}
\end{tiny}
\end{frame}


\begin{frame}{Section Summary}

  \begin{itemize}
    \item{Kokkos' primary build system is CMAKE.}
    \item{Kokkos options are transitively passed on, including many necessary compiler options.}
    \item{The Spack package manager does support Kokkos.}
    \item{If you write an application, and have few if any dependencies, building Kokkos as part of your code is an option with both CMake and GNU Makefiles.}
  \end{itemize}

\end{frame}
