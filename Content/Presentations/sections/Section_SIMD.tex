
%==========================================================================

\begin{frame}[fragile]

  {\Huge SIMD}

  \vspace{10pt}

  {\large Portable vector intrinsic types.}

  \vspace{20pt}

  \textbf{Learning objectives:}
  \begin{itemize}
    \item {How to use SIMD types to improve vectorization.}
    \item {SIMD Types as an alternative to ThreadVector loops.}
    \item {SIMD Types to achieve outer loop vectorization.}
  \end{itemize}

  \vspace{-20pt}

\end{frame}

%==========================================================================

\begin{frame}[fragile]{Vectorization In Kokkos}

   So far there were two options for achieving vectorization: 

\begin{itemize}
  \item{{\textbf{Hope For The Best}}: Kokkos semantics make loops inherently vectorizable, sometimes the compiler figures it even out.}
  \item{{\textbf{Hierarchical Parallelism}}: {\texttt{TeamVectorRange}} and {\texttt{ThreadVectorRange}} help the compiler with hints such as {\texttt{\#pragma ivdep}} or {\texttt{\#pragma omp simd}}}.
\end{itemize}

   \vspace{3pt}

  These strategies do run into limits though:

\begin{itemize}
  \item{Compilers often do not vectorize loops on their own.}
  \item{An optimal vectorization strategy would require \emph{outer-loop vectorization}.}
  \item{Vectorization with \texttt{TeamVectorRange} sometimes requires artifically introducing an additional loop level.}
\end{itemize}
\end{frame}

\begin{frame}[fragile]{Outer-Loop Vectorization}

   A simple scenario where for outer-loop vectorization:
	\vspace{-3pt}
  \begin{code}[linebackgroundcolor={},keywords={for,int}]
  for(int i=0; i<N; i++) {
    // expect K to be small odd 1,3,5,7 for physics reasons
    for(int k=0; k<K; k++) b(i) += a(i,k);
  }
  \end{code}

  Vectorization the \texttt{K}-Loop is not profitable:
  \begin{itemize}
	  \item{It is a short reduction.}
	  \item{Remainders will eat up much time.}
  \end{itemize}
	\vspace{5pt}

  Using \texttt{ThreadVectorRange} is cumbersome and requires split of \texttt{N}-Loop:

  \begin{code}[linebackgroundcolor={},keywords={parallel_for,for,int,TeamPolicy,ThreadVectorRange}]
parallel_for("VectorLoop",TeamPolicy<>(0,N/V,V),
  KOKKOS_LAMBDA ( const team_t& team ) {
  int i = team.league_rank() * V;
  for(int k=0; k<K; k++) 
    parallel_for(ThreadVectorRange(team,V), [&](int ii) {
      b(i+ii) += a(i+ii,k);
    });
});
    \end{code}
\end{frame}

\begin{frame}[fragile]{SIMD Types}
  
   To help with this situation and (in particular in the past) fix the lack of auto-vectorizing compilers \texttt{SIMD-Types} have been invented. They:
	\begin{itemize}
		\item{Are short vectors of scalars.}
		\item{Have operators such as \texttt{+=} so one can use them like scalars.}
		\item{Are compile time sized.}
		\item{Usually map directly to hardware vector instructions.}
	\end{itemize}
	
	\begin{block}{Important concept: SIMD Type}
		A SIMD variable is a \textbf{short vector} which acts like a scalar.
  \end{block}

	Using such a \texttt{simd} type one can simply achieve \emph{outer-loop} vectorization by using arrays of \texttt{simd} and dividing the loop range by its \emph{size}.
\end{frame}

\begin{frame}[fragile]{Outer-Loop Vectorization}

  Lets take a look back at the outer loop vectorization:
  \begin{code}[linebackgroundcolor={},keywords={for,int}]
  View<double*> b = ...
  View<double**> a = ...
  for(int i=0; i<N; i++) {
    // expect K to be small odd 1,3,5,7 for physics reasons
    for(int k=0; k<K; k++) b(i) += a(i,k);
  }
  \end{code}

  \pause
  Using SIMD types is conceptionally as simple as:
	\begin{itemize}
		\item Replace scalar type with SIMD type
		\item Adjust loop iteration count by SIMD length
	\end{itemize}

  \begin{code}[linebackgroundcolor={},keywords={for,int}]
  using simd_t = Kokkos::Experimental::simd<double, Kokkos::Experimental::simd_abi::native>;
  View<simd_t*> b = ...
  View<simd_t**> a = ...
  int V = simd_t::size();
  for(int i=0; i<N/V; i++) {
    // expect K to be small odd 1,3,5,7 for physics reasons
    for(int k=0; k<K; k++) b(i) += a(i,k);
  }
  \end{code}
\end{frame}

\begin{frame}[fragile]{C++23? SIMD}
	The ISO C++ standard has a \emph{Technical Specification} for \texttt{simd} (in \emph{parallelism v2}):

  \begin{code}[linebackgroundcolor={},keywords={template,class,public,using}]
template< class T, class Abi >
class simd {
public:
  using value_type = T;
  using reference = /* impl defined */;
  using abi_type = Abi;
  static constexpr size_t size();
  void copy_from(T const*, aligned_tag);
  void copy_to(T*, aligned_tag) const;
  T& operator[] (size_t);
  //Element wise operators
};

// Element Wise non-member operators

  \end{code}

\end{frame}

\begin{frame}[fragile]{C++23? SIMD ABI}
One interesting innovation here is the \texttt{Abi} parameter allowing for different, hardware specific, implementations.

	\vspace{8pt}

The most important in the proposal are:
\begin{itemize}
	\item{\textbf{scalar}: single element type.}
	\item{\textbf{native}: best fit for hardware.}
        \item{\textbf{fixed\_size$<$N$>$}: stores N elements. NOT YET IN KOKKOS.}
\end{itemize}

\pause
	\vspace{8pt}

	But \texttt{std::simd} is not in the standard yet, and doesn't support GPUs ...

	\pause
	\vspace{8pt}
	It also has other problems making it insufficient for our codes ...
\end{frame}

\begin{frame}[fragile]{Kokkos SIMD}
   Just at Sandia we had at least \textbf{5} different SIMD types in use.

   \vspace{8pt}
   A unification effort was started with the goal of:
   \begin{itemize}
      \item{Match the proposed \texttt{std::simd} API as far as possible.}
      \item{Support GPUs.}
      \item{Can be used stand-alone or in conjunction with Kokkos.}
      \item{Replaces all current implementations at Sandia for SIMD.}
   \end{itemize}

\end{frame}

\begin{frame}[fragile]{Kokkos SIMD}
  As with the C++23 SIMD type it takes Scalar argument and ABI
  \begin{code}
     template<class Scalar, class ABI>
     simd;
  \end{code}

  Two supported ABIs are:
  \begin{itemize}
     \item \texttt{simd\_abi::native}: whatever is best for the scalar type on the architecture
     \item \texttt{simd\_abi::scalar}: a single element
  \end{itemize}

  Aliasing with Scalar Views is possible (but careful with Layout and length!)
  \begin{code}
using simd_t = simd<double, simd::simd_abi::native>;
View<simd_t*> a("A",N);
View<double*> a_s(static_cast<double*>(a.data()),N*simd_t::size());
     
View<double*> b("B",M);
View<simd_t*> b_v(static_cast<simd_t*>(b.data()),M/simd_t::size());
  \end{code}

\end{frame}

\begin{frame}[fragile]{Exercise: Simple SIMD usage.}

  \textbf{Details}:
  \begin{small}
  \begin{itemize}
\item Location: \ExerciseDirectory{simd/Begin}
\item Include the \texttt{simd.hpp} header.
\item Change the data type of the views to use \texttt{simd$<$double, simd\_abi:native$>$}.
\item Create an unmanaged \texttt{View$<$double*$>$} of \texttt{results} using the \texttt{data()} function for the final reduction.  
\end{itemize}
  \end{small}

\begin{code}
   # Compile for CPU
   make -j KOKKOS_DEVICES=OpenMP
   # Compile for GPU
   make -j KOKKOS_DEVICES=Cuda
   # Run on GPU
   ./simd.cuda
\end{code}

	\vspace{-3pt}
\ul{\textbf{Things to try:}}
  \begin{small}
  \begin{itemize}
  \item Vary problem size (-N ...; -K ...)
  \item Compare behavior of scalar vs vectorized on CPU and GPU
  \end{itemize}
  \end{small}



\end{frame}

\begin{comment}
\begin{frame}[fragile]{The GPU SIMD Problem}
  The above exercise used a \textbf{scalar} simd type on the \textbf{GPU}.
  
	{\textbf Why wouldn't we use a fixed\_size instead?}

	\begin{itemize}
	   \item{Using a \texttt{fixed\_size} ABI will create a scalar of size \texttt{N} in each CUDA thread!}
	   \item{Loading a \texttt{fixed\_size} variable from memory would result in uncoalesced access.}
           \item{If you have correct layouts you get \texttt{outer-loop} vectorization implicitly on GPUs.}
	\end{itemize}

	\pause
	But what if you really want to use \textbf{warp}-level parallelization for SIMD types?

	\pause
	{\textbf We need \emph{two} SIMD types: a \emph{storage} type and a \emph{temporary} type!}
\end{frame}

\begin{frame}[fragile]{cuda\_warp ABI}
  \begin{block}{Important concept: simd::storage\_type}
    Every \texttt{simd$<$T,ABI$>$} has an associated \texttt{storage\_type} typedef.
  \end{block}

  To help with the GPU issue we split types between \textbf{storage} types used for \texttt{Views}, and \textbf{temporary} variables.

  \begin{itemize}
	  \item{Most \texttt{simd::simd} types will just have the same \texttt{storage\_type}.}
	  \item{\texttt{simd$<$T,cuda\_warp$<$N$> >$} will use warp level parallelism.}
	  \item{\texttt{simd$<$T,cuda\_warp$<$N$> >$::storage\_type} is different though!}
	  \item{Used in conjunction with \texttt{TeamPolicy}.}
  \end{itemize}
\end{frame}

\begin{frame}[fragile]{cuda\_warp ABI}
\textbf{Illustrating difference between \texttt{pack} and \texttt{cuda\_warp}}


      \begin{code}[linebackgroundcolor={},keywords={simd,parallel_for,using}]
using ABI = ... ;
View<simd<double,ABI::storage_type> A(...);
parallel_for(TeamPolicy<>(N,AUTO,V), 
 KOKKOS_LAMBDA(const teamt_t& team) {
  int i = team.league_rank()*team.team_size()+team.team_rank();
  simd<double,ABI> tmp = A(i);
});
       \end{code}
  \begin{columns}[]
    \begin{column}{.5\textwidth}
      \begin{code}[linebackgroundcolor={},keywords={pack,using}]
using ABI = pack<8>; int V=1;
\end{code}

       \includegraphics[width=0.75\textwidth]{figures/simd-fixedsize} 

    \end{column}
    \begin{column}{.5\textwidth}
      \begin{code}[linebackgroundcolor={},keywords={cuda_warp,using}]
using ABI = cuda_warp<8>; int V=8;
\end{code}

       \includegraphics[width=0.75\textwidth]{figures/simd-warp} 

    \end{column}
  \end{columns}
\end{frame}

\begin{frame}[fragile]{cuda\_warp ABI}

Example of using \texttt{storage\_type}:

\begin{code}[linebackgroundcolor={},keywords={template,class,simd,simd_storage,TeamPolicy,TeamThreadRange,public,using}]
// Using cuda_warp abi
using simd_t = simd::simd<T,simd::simd_abi::cuda_warp<V> >;
// Define simd_storage type
using simd_storage_t = simd_t::storage_type;
// Allocate memory
View<simd_storage_t**> data("D",N,M); // will hold N*M*V Ts

// Launch Loop with vectorlength V
parallel_for("Loop", TeamPolicy<>(N,M,V), 
  KOKKOS_LAMBDA(const team_t& team) {
    int i = team.league_rank();
    parallel_for(TeamThreadRange(team,M), [&](int j) {
      // Load storage type into internal type;
      simd_t tmp = data(i,j);
      // Do something with it
      tmp *= 2.0;
      // write values back
      data(i,j) = tmp;
      // or inline:
      // data(i,j) = 2.0*simd_t(data(i,j));
  }); 
});
\end{code}

\end{frame}


\begin{frame}[fragile]{Exercise: SIMD storage usage.}

  \textbf{Details}:
  \begin{small}
  \begin{itemize}
\item Location: \ExerciseDirectory{simd\_warp/Begin}
\item Include the \texttt{simd.hpp} header.
\item Change the data type of the views to use \texttt{simd::simd$<$double,simd::simd\_abi:cuda\_warp$<32>>$::storage\_type}.
\item Create an unmanaged \texttt{View$<$double*$>$} of \texttt{results} using the \texttt{data()} function for the final reduction.  
\item Use inside of the lambda the \texttt{simd::simd$<$double,simd::simd\_abi:cuda\_warp$<32>>$} as scalar type.
\end{itemize}
  \end{small}

\begin{code}
   # Compile for GPU
   make -j KOKKOS_DEVICES=Cuda
   # Run on GPU
   ./simd.cuda
\end{code}

\end{frame}
\end{comment}

\begin{frame}[fragile]{Advanced SIMD Capabilities}

Kokkos SIMD supports math operations:
\begin{itemize}
  \item{Common stuff like \texttt{abs}, \texttt{sqrt}, \texttt{exp}, ...}
\end{itemize}

\vspace{8pt}
It also supports masking:

	\begin{code}
// Scalar code with condition:
for(int i=0; i<N; i++) {
  if( a(i) < 100.0 ) b(i) = a(i);
  else b(i) = 100.0;
}

// Becomes
using simd_t = simd<double,simd_abi::native>;
using simd_mask_t = simd_t::mask_type;
   
for(int i=0; i<N/V; i++) {
  simd_t threshold(100.0), a_i(a_v(i));
  simd_mask_t is_smaller = threshold<a_i;

  b_v(i) = condition(is_smaller,a_i,threshold);
}
\end{code}
\end{frame}

\begin{frame}[fragile]{SIMD Summary}
	\begin{itemize}
		\item{SIMD types help vectorize code.}
		\item{In particular for \textbf{outer-loop} vectorization.}
		\item{There are \textbf{storage} and \textbf{temporary} types.}
		\item{Masking is supported too.}
	\end{itemize}
\end{frame}
%==========================================================================
