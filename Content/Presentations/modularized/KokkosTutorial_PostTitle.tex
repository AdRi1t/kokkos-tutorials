\usetheme{kokkos}

\newif\ifshort
\newif\ifmedium
\newif\iffull
\newif\ifnotoverview

\newcommand{\TutorialDirectory}{\texttt{Intro-Full}}
\newcommand{\ExerciseDirectory}[1]{\texttt{Exercises/#1/}}
\newcommand{\TutorialClone}{\texttt{Kokkos/kokkos-tutorials/\TutorialDirectory}}

\definecolor{darkgreen}{rgb}{0.0, 0.5, 0.0}
\definecolor{darkred}{rgb}{0.8, 0.0, 0.0}
\definecolor{orange}{rgb}{0.8, 0.33, 0.0}
\definecolor{purple}{rgb}{0.60, 0.20, 0.80}
\colorlet{bodyColor}{blue!20}
\colorlet{patternColor}{orange!30}
\colorlet{policyColor}{green!30}

% http://tex.stackexchange.com/questions/144448/color-a-text-line-in-a-code-lstlisting
\lstnewenvironment{code}[1][]%
{
  %with txfonts: OT1/txr/m/n/10
  %with default fonts: OT1/cmr/m/n/10
  %\fontfamily{cmr}\selectfont
  %\showthe\font
   \noindent
   \minipage{\linewidth}
   %\vspace{0.5\baselineskip}
   \lstset{mathescape, escapeinside={<@}{@>},
moredelim=**[is][{\btHL[fill=patternColor]}]{@pattern}{@pattern},
moredelim=**[is][{\btHL[fill=red!30]}]{@warning}{@warning},
moredelim=**[is][{\btHL[fill=policyColor]}]{@policy}{@policy},
moredelim=**[is][{\btHL[fill=bodyColor]}]{@body}{@body},
moredelim=**[is][{\btHL[fill=red!30]}]{@warning}{@warning},
moredelim=**[is][\color{black}]{@black}{@black},
moredelim=**[is][\color{blue}]{@blue}{@blue},
moredelim=**[is][\bf]{@bold}{@bold},
moredelim=**[is][\it]{@italic}{@italic},
moredelim=**[is][\color{boldblue}\bf]{@boldblue}{@boldblue},
moredelim=**[is][\color{red}]{@red}{@red},
moredelim=**[is][\color{green}]{@green}{@green},
moredelim=**[is][\color{gray}]{@gray}{@gray},
moredelim=**[is][\color{darkgreen}]{@darkgreen}{@darkgreen},
moredelim=**[is][\color{darkred}]{@darkred}{@darkred},
moredelim=**[is][\color{orange}]{@orange}{@orange},
moredelim=**[is][\color{purple}]{@purple}{@purple},
keywords={},
#1}
}
{
  \endminipage
  %\vspace{1.0\baselineskip}
}

\makeatletter
\newif\ifATOlinebackground
\lst@Key{linebackground}{\tiny}{\def\ATOlinebackground{#1}\global\ATOlinebackgroundtrue}
\makeatother

\lstnewenvironment{shell}[1][]{%
  \global\ATOlinebackgroundfalse
  \lstset{language=sh,%
    showstringspaces=false,
    aboveskip=0pt,
    frame=none,
    numbers=none,
    belowskip=2pt,
    breaklines=true,
    #1,
    }
  %\ifATOlinebackground
  \lstset{linebackgroundcolor={
    \ATOlinebackground
  }}
  %\fi
  }{}

\lstnewenvironment{cmake}[1][]{%
  \global\ATOlinebackgroundfalse
  \lstset{language=sh,%
    showstringspaces=false,
    aboveskip=0pt,
    frame=none,
    numbers=none,
    belowskip=2pt,
    breaklines=true,
    #1,
    }
  %\ifATOlinebackground
  \lstset{linebackgroundcolor={
    \ATOlinebackground
  }}
  %\fi
  }{}

\newcommand{\inlinecode}[1]{{\lstset{basicstyle=\ttfamily,keywordstyle={},showstringspaces=false}\lstinline$#1$}}
\newcommand{\inlineshell}[1]{{\lstset{basicstyle=\ttfamily,keywordstyle={},showstringspaces=false}\lstinline$#1$}}

\setbeamercolor{block title}{fg=white, bg=SandiaLightBlue}
\setbeamercolor{block body}{bg=lightgray}
\setbeamercolor{block title alerted}{fg=white, bg=SandiaRed}
\setbeamercolor{block body alerted}{bg=lightgray}



%\usepackage[texcoord,grid,gridunit=mm,gridcolor=red!10,subgridcolor=green!10]{eso-pic}
\usepackage[absolute,overlay]{textpos}





% http://tex.stackexchange.com/questions/8851/how-can-i-highlight-some-lines-from-source-code

\usepackage{pgf, pgffor}
\usepackage{listings}
\usepackage{lstlinebgrd} % see http://www.ctan.org/pkg/lstaddons

\makeatletter
%%%%%%%%%%%%%%%%%%%%%%%%%%%%%%%%%%%%%%%%%%%%%%%%%%%%%%%%%%%%%%%%%%%%%%%%%%%%%%
%
% \btIfInRange{number}{range list}{TRUE}{FALSE}
%
% Test in int number <number> is element of a (comma separated) list of ranges
% (such as: {1,3-5,7,10-12,14}) and processes <TRUE> or <FALSE> respectively

\newcount\bt@rangea
\newcount\bt@rangeb

\newcommand\btIfInRange[2]{%
    \global\let\bt@inrange\@secondoftwo%
    \edef\bt@rangelist{#2}%
    \foreach \range in \bt@rangelist {%
        \afterassignment\bt@getrangeb%
        \bt@rangea=0\range\relax%
        \pgfmathtruncatemacro\result{ ( #1 >= \bt@rangea) && (#1 <= \bt@rangeb) }%
        \ifnum\result=1\relax%
            \breakforeach%
            \global\let\bt@inrange\@firstoftwo%
        \fi%
    }%
    \bt@inrange%
}
\newcommand\bt@getrangeb{%
    \@ifnextchar\relax%
        {\bt@rangeb=\bt@rangea}%
        {\@getrangeb}%
}
\def\@getrangeb-#1\relax{%
    \ifx\relax#1\relax%
        \bt@rangeb=100000%   \maxdimen is too large for pgfmath
    \else%
        \bt@rangeb=#1\relax%
    \fi%
}

%%%%%%%%%%%%%%%%%%%%%%%%%%%%%%%%%%%%%%%%%%%%%%%%%%%%%%%%%%%%%%%%%%%%%%%%%%%%%%
%
% \btLstHL<overlay spec>{range list}
%
% TODO BUG: \btLstHL commands can not yet be accumulated if more than one overlay spec match.
%
\newcommand<>{\btLstHL}[2]{%
  \only#3{\btIfInRange{\value{lstnumber}}{#1}{\color{#2}\def\lst@linebgrdcmd{\color@block}}{\def\lst@linebgrdcmd####1####2####3{}}}%
}%
\makeatother






% http://tex.stackexchange.com/questions/15237/highlight-text-in-code-listing-while-also-keeping-syntax-highlighting
%\usepackage[T1]{fontenc}
%\usepackage{listings,xcolor,beramono}
\usepackage{tikz}

\makeatletter
\newenvironment{btHighlight}[1][]
{\begingroup\tikzset{bt@Highlight@par/.style={#1}}\begin{lrbox}{\@tempboxa}}
{\end{lrbox}\bt@HL@box[bt@Highlight@par]{\@tempboxa}\endgroup}

\newcommand\btHL[1][]{%
  \begin{btHighlight}[#1]\bgroup\aftergroup\bt@HL@endenv%
}
\def\bt@HL@endenv{%
  \end{btHighlight}%
  \egroup
}
\newcommand{\bt@HL@box}[2][]{%
  \tikz[#1]{%
    \pgfpathrectangle{\pgfpoint{1pt}{0pt}}{\pgfpoint{\wd #2}{\ht #2}}%
    \pgfusepath{use as bounding box}%
    \node[anchor=base west, fill=orange!30,outer sep=0pt,inner xsep=1pt, inner ysep=0pt, rounded corners=3pt, minimum height=\ht\strutbox+1pt,#1]{\raisebox{1pt}{\strut}\strut\usebox{#2}};
  }%
}
\makeatother



\usetikzlibrary{calc}
\usepackage{xparse}%  For \NewDocumentCommand

% tikzmark command, for shading over items
\newcommand{\tikzmark}[1]{\tikz[overlay,remember picture] \node (#1) {};}

\makeatletter
\NewDocumentCommand{\DrawBox}{s O{}}{%
    \tikz[overlay,remember picture]{
    \IfBooleanTF{#1}{%
        \coordinate (RightPoint) at ($(left |- right)+(\linewidth-\labelsep-\labelwidth,0.0)$);
    }{%
        \coordinate (RightPoint) at (right.east);
    }%
    \draw[red,#2]
      ($(left)+(-0.2em,0.9em)$) rectangle
      ($(RightPoint)+(0.2em,-0.3em)$);}
}

\NewDocumentCommand{\DrawBoxWide}{s O{}}{%
    \tikz[overlay,remember picture]{
    \IfBooleanTF{#1}{%
        \coordinate (RightPoint) at ($(left |- right)+(\linewidth-\labelsep-\labelwidth,0.0)$);
    }{%
        \coordinate (RightPoint) at (right.east);
    }%
    \draw[red,#2]
      ($(left)+(-\labelwidth,0.9em)$) rectangle
      ($(RightPoint)+(0.2em,-0.3em)$);}
}

\NewDocumentCommand{\DrawBoxWideBlack}{s O{}}{%
    \tikz[overlay,remember picture]{
    \IfBooleanTF{#1}{%
        \coordinate (RightPoint) at ($(left |- right)+(\linewidth-\labelsep-\labelwidth,0.0)$);
    }{%
        \coordinate (RightPoint) at (right.east);
    }%
    \draw[black,#2]
      ($(left)+(-\labelwidth,0.9em)$) rectangle
      ($(RightPoint)+(0.2em,-0.3em)$);}
}
\makeatother

\usetikzlibrary{positioning}

\usetikzlibrary{shapes}

\hypersetup{
    colorlinks=true,
    linkcolor=blue,
    filecolor=magenta,
    urlcolor=cyan,
}

