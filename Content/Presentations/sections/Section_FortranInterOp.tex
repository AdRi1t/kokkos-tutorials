\begin{frame}{Installation}
  \begin{itemize}
    \item Repository: \url{https://github.com/kokkos/kokkos-fortran-interop}
    \item Requirements:
      \begin{itemize}
        \item Kokkos 4.0 or newer
        \item C++17/Fortran08 compiler suites
      \end{itemize}
    \item Configure\\
      \texttt{cmake -DKokkos\_ROOT=/kokkos/path /interop/path}
  \end{itemize}
\end{frame}

\begin{frame}{What is Kokkos-Fortran-Interop?}
  \begin{itemize}
    \item Kokkos-Fortran offers wrappers around:
      \begin{itemize}
        \item \texttt{Kokkos::initialize(argc, argv)}
        \item \texttt{Kokkos::finalize()}
        \item \texttt{Kokkos::print\_configuration(output)}
        \item \texttt{Kokkos::View} 
        \item \texttt{Kokkos::DualView}
      \end{itemize}
    \item User kernels are written in C\texttt{++} $\rightarrow$ need to use
      \texttt{iso\_c\_binding}
    \item Only a subset of Kokkos capabilities are exposed
  \end{itemize}
\end{frame}

\begin{frame}{Starting a program}
  \begin{itemize}
    \item Similar to MPI, Kokkos needs to be initialized by calling
      \texttt{kokkos\_initialize} and finalized by calling \texttt{kokkos\_finalize}
    \item The \texttt{kokkos\_initialize} subroutine initializes Kokkos and
      reads command line arguments. It should be called after
      \texttt{MPI\_Initialize}
    \item \texttt{kokkos\_print\_configure("output.txt")} prints the
      Kokkos configuration to the file \texttt{output.txt}
  \end{itemize}
\end{frame}

\begin{frame}[containsverbatim]{Simple Example}
  \begin{minted}{fortran}
program my_kokkos_code
  use :: flcl_util_kokkos_mod

  ! Initialize Kokkos
  ! This subroutine reads command line arguments
  call kokkos_initialize()

  ! Print the configuration in a file
  call kokkos_print_configure('kokkos.out')

  ! Finalize Kokkos
  call kokkos_finalize()
end program my_kokkos_code
  \end{minted}
\end{frame}

\begin{frame}{Kokkos::View}
  \begin{itemize}
    \item \texttt{Kokkos::View} is Kokkos equivalent to an array
    \item \texttt{Kokkos::View} can have up to 8 dimensions in C++ but they are
      limited to 7 dimensions in Fortran due to limitation of the library
    \item Supported Fortran types: logical, 32-bit integer, 64-bit
      integer, 32-bit real, 64-bit real, 32-bit complex, 64-bit complex, and
      index (positive 64-bit integer)
    \item Types follow the pattern \texttt{view\_<type>\_<dimension>\_t}, e.g.,
      \texttt{view\_r64\_1d\_t} is a one-dimensional view of 64-bit real
  \end{itemize}
\end{frame}

\begin{frame}{Kokkos::View}
  \begin{itemize}
    \item The memory space defines where the memory is allocated
    \item Supported memory spaces: \texttt{Kokkos::HostSpace},
      \texttt{Kokkos::CudaManagedSpace}, and \texttt{Kokkos::HIPManagedSpace}
    \item The memory space is determined during the configuration of the
      Kokkos-Fortran-Interop library, based on the memory space
      configuration of Kokkos.
  \end{itemize}
\end{frame}

\begin{frame}{Kokkos::View}
  \begin{itemize}
    \item \texttt{Kokkos::View} can be allocated directly or built from an
      array. In the latter case, the \texttt{Kokkos::View} can only be used on
      the host % I don't see the point of this
    \item \texttt{Kokkos::View} that are allocated with
      \texttt{kokkos\_allocate\_view} must also be deallocated with
      \texttt{kokkos\_deallocate\_view}
    \item \texttt{kokkos\_allocate\_view} initializes all the elements of the
      \texttt{Kokkos::View} to zero
    \item \texttt{Kokkos::View} cannot be accessed directly from Fortran instead
      it can be accessed through a \texttt{pointer}
  \end{itemize}
\end{frame}

\begin{frame}[containsverbatim]{Kokkos::View}
  \begin{minted}{fortran}
use, intrinsic :: iso_c_binding
use, intrinsic :: iso_fortran_env
use :: flcl_mod

! Kokkos View only accessible from C++
type(view_r64_1d_t) :: v_c_y
! Pointer to access the Kokkos View from Fortran
real(real64), pointer, dimension(:) :: c_y
integer :: mm = 5000

call kokkos_allocate_view(c_y, v_c_y, 'c_y', &
                          int(mm, c_size_t))

! Do stuff

call kokkos_deallocate_view(c_y, v_c_y)
  \end{minted}
\end{frame}

\begin{frame}{Kokkos::DualView}
  \begin{itemize}
    \item \texttt{Kokkos::DualView} are similar to \texttt{Kokkos::View} but
      they are composed of two \texttt{Kokkos::View}s, one on the host and one on
      the device.
    \item It is the user's responsibility to synchronize the data
    \item The synchronization must be done in C++
    \item Supported memory spaces: \texttt{Kokkos::HostSpace},
      \texttt{Kokkos::Cuda}, \texttt{Kokkos::HIP}, and \texttt{Kokkos::SYCL}
    \item \texttt{Kokkos::DualView} cannot be accessed directly from Fortran instead
      it can be accessed through a \texttt{pointer}
  \end{itemize}
\end{frame}

\begin{frame}[containsverbatim]{Kokkos::DualView}
  \begin{minted}{fortran}
use, intrinsic :: iso_c_binding
use, intrinsic :: iso_fortran_env
use :: flcl_mod

real(real64), pointer, dimension(:) :: c_y
type(dualview_r64_1d_t) :: v_c_y
integer :: mm = 5000

call kokkos_allocate_dualview(c_y, v_c_y, 'c_y', &
                              int(mm, c_size_t))

! Do stuff

call kokkos_deallocate_dualview(c_y, v_c_y)
  \end{minted}
\end{frame}

\begin{frame}{Kernel}
  \begin{itemize}
    \item Kernels using Kokkos are written in C++
    \item Use C-binding from Fortran standard 
    \item Create a subroutine that calls the C++ function
  \end{itemize}
\end{frame}

\begin{frame}[containsverbatim]{Initialize View}
  \begin{minted}{fortran}
use, intrinsic :: iso_c_binding
use, intrinsic :: iso_fortran_env
use :: flcl_mod
use :: my_init_mod

real(real64), pointer, dimension(:) :: c_y
type(view_r64_1d_t) :: v_c_y
integer :: mm = 5000

call kokkos_allocate_view(c_y, v_c_y, 'c_y', &
                          int(mm, c_size_t))

call my_init(v_c_y)

call kokkos_deallocate_view(c_y, v_c_y)
  \end{minted}
\end{frame}

\begin{frame}[containsverbatim]{Initialize View}
  \begin{minted}{fortran}
module my_init_mod
    use, intrinsic :: iso_c_binding
    use, intrinsic :: iso_fortran_env
    use :: flcl_mod
    implicit none
    public
      interface
        subroutine my_f_init( y ) &
          & bind(c, name='my_c_init')
          import
          type(c_ptr), intent(in) :: y
        end subroutine my_f_init
      end interface
  \end{minted}
\end{frame}

\begin{frame}[containsverbatim]{Initialize View}
  \begin{minted}{fortran}
      contains

        subroutine my_init( y )
          type(view_r64_1d_t), intent(inout) :: y
          call my_f_init(y%ptr())
        end subroutine my_init
  
end module my_init_mod
  \end{minted}
\end{frame}

\begin{frame}[containsverbatim]{Initialize View}
  \begin{minted}{C++}
#include <Kokkos_Core.hpp>
#include <flcl-cxx.hpp>
using view_type = flcl::view_r64_1d_t;

extern "C" {
  void my_c_init(view_type **v_y) {
    view_type y = **v_y;
    Kokkos::parallel_for(
        "init", y.extent(0), 
        KOKKOS_LAMBDA(int idx) {y(idx) += idx;});
    Kokkos::fence();
  }
}
  \end{minted}
\end{frame}

\begin{frame}{Exercise}
  \begin{itemize}
    \item Use \texttt{Kokkos::View} to do an \emph{axpy}
    \item Do not forget to install the library
  \end{itemize}
\end{frame}

