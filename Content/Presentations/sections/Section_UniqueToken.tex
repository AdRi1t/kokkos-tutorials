%==========================================================================

\begin{frame}[fragile]{}

  {\Huge{Unique Token}}

  \vspace{20pt}

  \textbf{Learning objectives:}
  \begin{itemize}
    \item {Understand concept of unique tokens and thread-safe resource access. }
    \item {Learn how to acquire per-team unique ids.}
    \item {Understand the difference between \textbf{Global} and \textbf{Instance} scope.}
  \end{itemize}

  \vspace{-20pt}

\end{frame}

%==========================================================================

\begin{frame}[fragile]{Unique Tokens - Motivation}
\textbf{Why do we need a unique token concept?}
\begin{itemize}
\item{Within Functor operator / Lambda there is no portable way to identify the active execution resource (thread id)}
\item{Some algorithms make efficient use of shared resources by dividing based on execution resource (thread id)}
\item{Thread Id is not consistent or portable across all execution environments}
\item{Unique Token provides consistent identifier for resource allocations and work division}
\end{itemize}

\end{frame}

%==========================================================================

\begin{frame}[fragile]{Unique Tokens - Motivation}
\textbf{Original Example: Random Number Generator Pool}

\begin{code}
int N = 10000000
int K = ...;
RandomGenPool pool(K,seed);
parallel_for("Loop", N, KOKKOS_LAMBDA(int i) {
  int gen_id = ...
  auto gen = pool[gen_id];
});
\end{code}

\vspace{5pt}
\textbf{How many generators do we need (\texttt{K})?}

\pause
\vspace{5pt}
\textbf{How to get a unique one in the loop (\texttt{gen\_id})?}

\pause
\vspace{5pt}
In OpenMP we could use the \textbf{thread-id} but what in CUDA?

\end{frame}

%==========================================================================

\begin{frame}[fragile]{Unique Tokens - Motivation}
\vspace{-10pt}
\textbf{Motivating Example}

\vspace{5pt}
\textbf{OpenMP}
    \begin{code}[frame=single, keywords={}]
int K = omp_get_max_threads();
Kokkos::parallel_for("L", N, KOKKOS_LAMBDA(int i) {
  int tid = omp_get_thread_num();
});
    \end{code}

\vspace{5pt}
\textbf{CUDA}
    \begin{code}[frame=single, keywords={}]
int K = N; // ??
Kokkos::parallel_for("L", N, KOKKOS_LAMBDA(int i) {
  int tid = threadIdx.x + blockDim.x * blockIdx.x; //i??
});
    \end{code}

\pause
  \textbf{{\color{red}Problem}}: In \textbf{CUDA} there is no way to get \textbf{hardware thread-id}.

  \vspace{5pt}
  \pause

  \textbf{Solution:} We need a thread-safe and portable way to obtain unique identifier that is per-thread specific.

  \vspace{5pt}

  \hspace{20pt}{\Large $\Rightarrow$ \textbf{UniqueToken}}

  \vspace{-5pt}

\end{frame}

%==========================================================================

\begin{frame}[fragile]{Unique Token}

\textbf{UniqueToken is a pool of IDs}

\begin{itemize}
\item User acquires an ID and releases it again.
\end{itemize}

   \begin{code}[frame=single, keywords={UniqueToken,size,acquire,release}]
    UniqueToken<ExecutionSpace> token;
    int number_of_uniqe_ids = token.size();
    RandomGenPool pool(number_of_unique_ids,seed);
    parallel_for("L", N, KOKKOS_LAMBDA(int i) {
      int id = token.acquire();
      RandomGen gen = pool(id);
      ...
      token.release(id);
    });
   \end{code}

\pause
\begin{itemize}
\item Do not acquire more than one token in an iteration.
\item You must release the token again.
\item By default the range of ids is 0 to \texttt{ExecSpace().concurrency()}.
\end{itemize}

\end{frame}

%==========================================================================

\begin{frame}[fragile]{Unique Token - Global vs. Instance Scope}

\textbf{Sometimes you need a Global UniqueToken}
\begin{itemize}
\item Submitting concurrent kernels to CUDA streams (Module 5)
\item Shared resource in a multi-threaded environment like Legion
\end{itemize}

\vspace{-5pt}
\pause
\begin{block}{UniqueToken is Scoped}
UniqueToken has a Scope template parameter which by default is 'Instance' but can be 'Global'.
\end{block}

\pause
   \begin{code}[frame=single, keywords={UniqueTokenScope,Global,stream1,stream2}]
void foo() {
  UniqueToken<ExecSpace,UniqueTokenScope::Global> token_foo;
  parallel_for("L", RangePolicy<ExecSpace>(stream1,0,N)
    , functor_a(token_foo));
}
void bar() {
  UniqueToken<ExecSpace,UniqueTokenScope::Global> token_bar;
  parallel_for("L", RangePolicy<ExecSpace>(stream2,0,N)
    , functor_b(token_bar));
}
   \end{code}

\pause
\texttt{token\_foo} and \texttt{token\_bar} will provide non-conflicting ids.
\end{frame}
%==========================================================================

\begin{frame}[fragile]{Unique Token - Per Team}

  \textbf{UniqueToken can also be used for Per-Team resources}

\vspace{5pt}
  There are less teams active than threads. How to get an ID?

\vspace{5pt}
\pause
\begin{block}{Sized UniqueToken}
{UniqueToken supports custom ranges of ids via constructing sized tokens.}
\end{block}

\pause
\textbf{Acquiring a per-team unique id requires three steps:}
\begin{itemize}
  \item Compute the range via \texttt{concurrency} and \texttt{team\_size}.
  \item Create a sized \texttt{UniqueToken}.
  \begin{itemize}
    \item For performance reason make it a bit larger than necessary.
  \end{itemize}
  \item Acquire and broadcast a token in a \texttt{single} pattern.
\end{itemize}
\end{frame}

\begin{frame}[fragile]{Unique Token - Per Team}

   \begin{code}[frame=single, keywords={team_size,concurrency,token,single,PerTeam}]
// Figure out the team size
int team_size = ...;
// How many teams are actually in-flight
int num_active_teams = ExecSpace().concurrency()/team_size;
// Create the token
UniqueToken<ExecSpace> token(num_active_teams * 1.2);

parallel_for("L", TeamPolicy<ExecSpace>(N,team_size),
  KOKKOS_LAMBDA(const team_t& team) {
    int id;
    // Acquire an id and broadcast it with a single thread
    single(PerTeam(team),[&](int &lid) {
      lid = token.acquire();
    },id);
    ...
    // Release the id again (likely you want a barrier first!)
    single(PerTeam(team),[&]() {
      token.release(id);
    });
   \end{code}


\end{frame}

%==========================================================================


\begin{frame}[fragile]{Exercise UniqueToken}

  \begin{small}
  \begin{itemize}
  \item Location: \ExerciseDirectory{unique\_token/Begin}
  \item Assignment: Convert scatter\_add\_loop to use \texttt{UniqueToken}, removing \#ifdef's
  \item Compile and run on both CPU and GPU
  \end{itemize}
  \end{small}

\begin{code}
make -j KOKKOS_DEVICES=OpenMP  # CPU-only using OpenMP
make -j KOKKOS_DEVICES=Cuda    # GPU - note UVM in Makefile 
# Run exercise
./uniquetoken.host
./uniquetoken.cuda
# Note the warnings, set appropriate environment variables
\end{code}

  \begin{scriptsize}
  \begin{itemize}
  \item Compare performance on CPU of the three variants
  \item Compare performance on GPU of the two variants
  \item Vary problem size: first and second optional argument
  \end{itemize}
  \end{scriptsize}

\end{frame}

%==========================================================================

\begin{frame}{Section Summary}

  \begin{itemize}
    \item{\textbf{UniqueToken} provides a thread safe portable way to divide thread or team specific resources}
    \item{\textbf{UniqueToken} can be sized, such that it returns only ids within a specific range.}
    \item{A \textbf{Global} scope UniqueToken can be acquired, allowing safe ids accross disjoint concurrent code sections.}
  \end{itemize}

\end{frame}

